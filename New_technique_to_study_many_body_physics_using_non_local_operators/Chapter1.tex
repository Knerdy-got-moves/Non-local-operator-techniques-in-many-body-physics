\pagenumbering{arabic} %necessary to start numbering pages in arabic numerals from here
\setcounter{page}{1} 


\chapter{History and Introduction}
 
\section{History}
The inspiration behind this study starts with the discovery of the Luttinger liquid model, proposed by S. Tomonaga in 1950\cite{Tom50}. N. N. Bogolyubov inspired it in 1947, while he was trying to explain superfluidity, where he used expressions for bosons in terms of fermions to represent the Hamiltonian as a product of two boson operators\cite{Bog47}.  The model showed that second-order interactions between electrons could be modelled as bosonic interactions in a one-dimensional multi-fermionic system under certain constraints. In 1963, J.M. Luttinger reformulated the theory in terms of Bloch sound waves. Through this, an exactly soluble model of a one‐dimensional many‐fermion system was proposed\cite{Lut63}. Soon after, in 1965, D.C. Mattis and E. H. Liebeld observed that charge density $\rho({\bf{p}})$ was ipso facto associated with the Fermi‐Dirac field. They then used this observation to solve for the and obtain the exact (and nontrivial) energy spectrum, free energy, and dielectric constant. The Luttinger liquid model was also extended to more realistic interactions in one dimension\cite{DCMEHL65}. In 1992, F.D.M. Haldane proposed the of Luttinger's theorem and bosonization of the Fermi surface. This was an important work in this field, using the previously developed ideas and giving a better understanding through bosonization in one dimension\cite{Hal83}. In 1994, A. H. Castro Neto and Eduardo Fradkin bosonized the low-energy excitations of Fermi liquids in d-dimensions in the limit of long wavelengths. The bosons were a coherent superposition of electron-hole pairs and were related with the displacements of the Fermi surface in some arbitrary direction. They constructed a coherent-state path integral for the "bosonized" theory and showed that it represented histories of the shape of the Fermi surface. The Landau theory of Fermi liquids could be obtained from this formalism in the absence of the nesting of the Fermi surface and singular interactions\cite{Fra94}. 

\section{Introduction}
We start with the results of the great physicists mentioned above and want to propose a new formalism that can tackle the harder questions like finding the many-particle Green's function for a system with many correlated degrees of freedom. These methods proposed are valuable to progress further research in physics as opposed to using just perturbative methods as they may have more scope in predicting new physical phenomena rather than explaining the current phenomena through perturbative means for taking into consideration interactions in fermionic systems.

For this, G. Setlur in 1995, proposed the idea of Sea-displacement operators in the fermionic subspace. These operators form the basis of this formalism\cite{Gir98}.

The definitions for these operators are given in the respective chapter 3, Eq. \eqref{Eq.3.12},\eqref{Eq.3.13}\eqref{Eq.3.14} and \eqref{Eq.3.15}.

In principle, these operators displace a fermion from below the fermi level to above it, hence creating a hole-particle pair. Thus, it forms the heart of our physics.
\begin{figure}
    \centering
    \includegraphics[width=0.5\linewidth]{Figures/Chapter_1_figures/adagger pq.png}
    \caption{Diagram illustrating the process described by Eq. \eqref{Eq.3.14} The Fermi level is represented by the horizontal dashed line. The lower point represents the state $(p - q/2, <)$, and the upper point represents the state $(p + q/2, >)$. The arrow indicates the action of destroying a particle below the Fermi level and creating one above it, labeled as $a^\dagger_p(q)$. }
    \label{adagpq}
\end{figure}
\begin{figure}
    \centering
    \includegraphics[width=0.5\linewidth]{Figures/Chapter_1_figures/apq.png}
    \caption{Diagram illustrating the process described by Eq. \eqref{Eq.3.15} The Fermi level is represented by the horizontal dashed line. The lower point represents the state $(p - q/2, <)$, and the upper point represents the state $(p + q/2, >)$. The arrow indicates the action of destroying a particle below the Fermi level and creating one above it, labeled as $a_p(q)$.}
    \label{apq}
\end{figure}
