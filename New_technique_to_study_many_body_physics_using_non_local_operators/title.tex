\begin{titlepage}
\begin{center}
         %this is to have the logo
\begin{figure}
                     \centering
                     \includegraphics[width=0.35\linewidth]{IITG_logo.png}
                    
                 \end{figure}
                 
        \Huge
        \textbf{New technique to study many-body systems using non-local operators}\\
        
     
        \vspace{1.5cm}
     \Large   
       {\bf Rishi Paresh Joshi}\\ \large{3rd Year, Integrated MSc, Physics\\ National Institute of Science Education and Research Bhubaneswar} %add any letters that should follow like B.Sc. or M.Sc. or AMRSC, etc

        
        
        \vfill
        
        
            
        \Large 
        \textbf{Under the supervision of \\ Dr. Girish Sampath Setlur}\\
        \large
        Physics Department\\
        IIT Guwahati\\
        \vfill
        Summer, 2024 \\
        
            
    \end{center}

\end{titlepage}
\newpage
\section{Acknowledgements}
\par I would like to express my sincere gratitude towards my guide, Prof. Girish Sampath Setlur, for his guidance and expertise throughout this internship, which has been invaluable in enhancing my understanding of this fascinating field. This experience has enriched my knowledge and strengthened my passion for research. I thank Sir, for his patience in seeing that I understand the topic and am on par with the research progress. Under his guidance, I have made valuable additions to my skill set.
I thank him for this wonderful experience and look forward to contributing and taking this research ahead.
\newpage
\section{Abstract}
This report briefly discusses the basics of the cutting-edge formalism developed to study the field of condensed matter in understanding interactions. This report very briefly introduces the past work in the Luttinger liquid model by authors like N. N. Bogolyubov \cite{Bog47}, S. Tomonaga \cite{Tom50}, J.M. Luttinger \cite{Lut63}, D. C. Mattis; E. H. Lieb \cite{DCMEHL65} and F. D. M. Haldane\cite{Hal83}. The main focus is on a new method to study many-body physics, using operators in the Fock space known as Sea-displacement operators\cite{Gir98}. While a lot has been done, some work is still in progress; G. S. Setlur uses this formalism using Sea-displacement operators to calculate the Greens function for a many-body system at finite temperature in many dimensions. This report introduces the creation and annihilation operators in second quantization and defines Fock space. Once we are done with the basics of quantum mechanics, we move towards defining the Sea-displacement operators and some important operators in terms of the Sea-displacement operators. We derive the Fermi Dirac distribution using this formalism. We conclude with the results and future works in this exciting field.