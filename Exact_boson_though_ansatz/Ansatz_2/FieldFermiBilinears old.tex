\documentclass{article}
\usepackage{mathtools}
\usepackage{cancel}
\usepackage{amsmath}
\usepackage[colorlinks]{hyperref}
\usepackage{amssymb}
\usepackage{graphicx}
\usepackage{float}
\usepackage{multicol}
\usepackage[top=1cm,left=1cm,right=1cm]{geometry}
\usepackage{chngpage}
\newcommand{\tab}[1]{\hspace{.2\textwidth}\rlap{#1}}
\usepackage{hyperref}
\usepackage{comment}
%\usepackage{bibtopic}
%\usepackage[utf8]{inputenc}
\usepackage[english]{babel}
%\newcommand{\tab}[1]{\hspace{.2\textwidth}\rlap{#1}}
\usepackage{tikz}
\newcounter{markeq}
\setcounter{markeq}{0}
\newcommand*\bmarkeq{%
   \stepcounter{markeq}%
   \tikz[remember picture]\node(startframe-\themarkeq){\strut};}
\newcommand*\emarkeq{%
   \begin{tikzpicture}[remember picture,overlay]
     \node (endframe-\themarkeq){\strut};
     \draw[,red,opacity=2] (startframe-\themarkeq.north) rectangle (endframe-\themarkeq.south);
   \end{tikzpicture}%
}

\begin{document}

\title{ Field operator in terms of Fermi Bilinears-2 }
\date{}
\author{Girish S. Setlur, rewritten by Rishi Paresh Joshi}

\maketitle
\small
 


\section{  Fermi-bilinears in terms of $ a_{ {\bf{k}} }({\bf{q}}) $ }

 
\[
c^{\dagger}_{ {\bf{p}} + {\bf{q}}/2, > }c_{ {\bf{p}}-{\bf{q}}/2, < } \mbox{     }  = \mbox{           } a^{\dagger}_{ {\bf{p}} }({\bf{q}})
\]


\[
c^{\dagger}_{ {\bf{p}} + {\bf{q}}/2, < }c_{ {\bf{p}}-{\bf{q}}/2, > } \mbox{     }  = \mbox{           }
a_{ {\bf{p}} }(-{\bf{q}})
\]


\[
c^{\dagger}_{ {\bf{p}} + {\bf{q}}/2, < }c_{ {\bf{p}}-{\bf{q}}/2, < } \mbox{     }  = \mbox{           }
n_F({\bf{p}})\mbox{  }\delta_{ {\bf{q}}, 0 }
 - \sum_{ {\bf{q}}_1 } \frac{1}{N_{>}} \mbox{  } a^{\dagger}_{ {\bf{p}}-{\bf{q}}/2 + {\bf{q}}_1/2 }({\bf{q}}_1)
a_{ {\bf{p}}+{\bf{q}}_1/2 }(-{\bf{q}}+{\bf{q}}_1)
\]


\[
c^{\dagger}_{ {\bf{p}} + {\bf{q}}/2, > }c_{ {\bf{p}}-{\bf{q}}/2, > } \mbox{     }  = \mbox{           }
\sum_{ {\bf{q}}_1 } \frac{1}{N_{>}} \mbox{  } a^{\dagger}_{ {\bf{p}}+{\bf{q}}/2 - {\bf{q}}_1/2 }({\bf{q}}_1)
a_{ {\bf{p}}-{\bf{q}}_1/2 }(-{\bf{q}}+{\bf{q}}_1)
\]

---------------------------------------------------------------------------------------------------------------------

\[
[a_{ {\bf{P}} }({\bf{Q}}),N_{>}]
\mbox{             }  = \mbox{           }  a_{ {\bf{P}} }({\bf{Q}})
\]


\[
[a^{\dagger}_{ {\bf{P}} }({\bf{Q}}),N_{>}]
\mbox{             }  = \mbox{           }  - a^{\dagger}_{ {\bf{P}} }({\bf{Q}})
\]



\[
f( N_{ > } )\mbox{          }
a_{ {\bf{P}} }({\bf{Q}})=a_{ {\bf{P}} }({\bf{Q}}) \mbox{       }f( N_{>} -1 )
\]



\[
a^{\dagger}_{ {\bf{P}} }({\bf{Q}})\mbox{       } f(N_{>})
= f( N_{>} -1 ) \mbox{       }a^{\dagger}_{ {\bf{P}} }({\bf{Q}})
\]


----------------------------------------------------------------------------------------------------------------

\[
c^{\dagger}_{ {\bf{p}} + {\bf{q}}/2, > }c_{ {\bf{p}}-{\bf{q}}/2, < } \mbox{     }  = \mbox{           } a^{\dagger}_{ {\bf{p}} }({\bf{q}})
\]


\[
c^{\dagger}_{ {\bf{p}} + {\bf{q}}/2, < }c_{ {\bf{p}}-{\bf{q}}/2, > } \mbox{     }  = \mbox{           }
a_{ {\bf{p}} }(-{\bf{q}})
\]

If $ {\bf{q}} \neq 0 $ then,
\[
c_{ {\bf{p}}-{\bf{q}}/2, < }c^{\dagger}_{ {\bf{p}} + {\bf{q}}/2, < }  \mbox{     }  = \mbox{           } 
  \sum_{ {\bf{q}}_1 }  a^{\dagger}_{ {\bf{p}}-{\bf{q}}/2 + {\bf{q}}_1/2 }({\bf{q}}_1)\mbox{  }
  \frac{1}{N_{>}+1} \mbox{  }
a_{ {\bf{p}}+{\bf{q}}_1/2 }(-{\bf{q}}+{\bf{q}}_1)
\]


\[
c^{\dagger}_{ {\bf{p}} + {\bf{q}}/2, > }c_{ {\bf{p}}-{\bf{q}}/2, > } \mbox{     }  = \mbox{           }
\sum_{ {\bf{q}}_1 }  a^{\dagger}_{ {\bf{p}}+{\bf{q}}/2 - {\bf{q}}_1/2 }({\bf{q}}_1)\mbox{  }
\frac{1}{N_{>}+1} \mbox{  }
a_{ {\bf{p}}-{\bf{q}}_1/2 }(-{\bf{q}}+{\bf{q}}_1)
\]

Also we define,

\[
c_{ {\bf{p}}, < }c^{\dagger}_{ {\bf{p}}, < }  \mbox{     }  = \mbox{           }
lim_{ {\bf{q}} \rightarrow 0 } \mbox{     } c_{ {\bf{p}}-{\bf{q}}/2, < }c^{\dagger}_{ {\bf{p}} + {\bf{q}}/2, < }  
\]


\[
c^{\dagger}_{ {\bf{p}}, > }c_{ {\bf{p}}, > } \mbox{     }  = \mbox{           }
lim_{ {\bf{q}} \rightarrow 0 } \mbox{     }c^{\dagger}_{ {\bf{p}} + {\bf{q}}/2, > }c_{ {\bf{p}}-{\bf{q}}/2, > } 
\]

---------------------------------------------------------------------------------------------------------------------
 
 


\section{ Field operator }

Set,
 \[
e^{ i N \theta_{ {\bf{p}} } }e^{ i N \theta_{ {\bf{p}}^{'} } } \mbox{            } 
= \mbox{           } 0 
 \]
and
 \[
e^{ i N \theta_{ {\bf{p}} } }e^{ -i N \theta_{ {\bf{p}}^{'} } } \mbox{            } 
= \mbox{           } \delta_{ {\bf{p}}, {\bf{p}}^{'} }
 \]
 This means,
\begin{equation}
\label{cp>}
  c_{ {\bf{p}}, > }  =  \sum_{ {\bf{q}}_2 }
 \mbox{     } e^{- i N \theta_{ {\bf{p}}-{\bf{q}}_2 } }
 \mbox{        }
 c_{ {\bf{p}}-{\bf{q}}_2,< } \mbox{            } \frac{1}{\sqrt{n_{{\bf{p}}-{\bf{q}}_2,<}}}  \mbox{   }
  \frac{1}{\sqrt{N_{>}+1}}  \mbox{  }a_{ {\bf{p}} - {\bf{q}}_2/2 }({\bf{q}}_2)  
\end{equation}

and
\begin{equation}
\label{c^+p<}
 c^{\dagger}_{ {\bf{p}}, < }    \mbox{            }  =    \mbox{            } \sqrt{N_{>}+1}\mbox{          }\frac{1}{\sqrt{n_{{\bf{p}},<}}} \mbox{        }
c^{\dagger}_{ {\bf{p}}, < } \mbox{  } e^{ i N \theta_{ {\bf{p}} } }
\mbox{       } 
+ \sum_{  {\bf{q}}_1 } \frac{1}{\sqrt{n_{{\bf{p}}+{\bf{q}}_1,>}}}\mbox{       } 
  c^{\dagger}_{ {\bf{p}}+{\bf{q}}_1,> } \mbox{          }\frac{1}{\sqrt{N_{>}+1}}
\mbox{  } e^{- i N \theta_{ {\bf{p}}+{\bf{q}}_1 } }\mbox{          }
 a_{ {\bf{p}}+ {\bf{q}}_1/2 }({\bf{q}}_1)   
\end{equation}



---------------------------------------------------------------------------------------

{\bf{Time evolution of the non-interacting system:}}

Note that for free fermions,
\[
c_{ {\bf{p}},s }(t) \mbox{         } = \mbox{            } 
c_{ {\bf{p}},s} \mbox{            } e^{ - i \epsilon_p \mbox{  } t }
\]
and
\[
a_{ {\bf{k}} }({\bf{q}};t)  \mbox{         } = \mbox{            } e^{ - i \frac{ {\bf{k.q}} }{m} \mbox{  } t } 
\mbox{            }a_{ {\bf{k}} }({\bf{q}}) 
\]
This means,
\[
c_{ {\bf{p}}, > }(t)  = \sum_{ {\bf{q}}_2 }
 \mbox{     } e^{- i N \theta_{ {\bf{p}}-{\bf{q}}_2 } }
 \mbox{        }
 c_{ {\bf{p}}-{\bf{q}}_2,< }(t) \mbox{            } \frac{1}{\sqrt{n_{{\bf{p}}-{\bf{q}}_2,<}}}  \mbox{   }
  \frac{1}{\sqrt{N_{>}+1}}  \mbox{  }a_{ {\bf{p}} - {\bf{q}}_2/2 }({\bf{q}}_2;t)
\]
and
\[
c^{\dagger}_{ {\bf{p}}, < }(t)    \mbox{            }  =    \mbox{            } \sqrt{N_{>}+1}\mbox{          }\frac{1}{\sqrt{n_{{\bf{p}},<}}} \mbox{        }
c^{\dagger}_{ {\bf{p}}, < }(t) \mbox{  } e^{ i N \theta_{ {\bf{p}} } }
\mbox{       } 
+ \sum_{  {\bf{q}}_1 } \frac{1}{\sqrt{n_{{\bf{p}}+{\bf{q}}_1,>}}}\mbox{       } 
  c^{\dagger}_{ {\bf{p}}+{\bf{q}}_1,> }(t) \mbox{          }\frac{1}{\sqrt{N_{>}+1}}
\mbox{  } e^{- i N \theta_{ {\bf{p}}+{\bf{q}}_1 } }\mbox{          }
 a_{ {\bf{p}}+ {\bf{q}}_1/2 }({\bf{q}}_1;t)
\]
Note that,
\[
 c_{ {\bf{p}}-{\bf{q}}_2,< }(t) \mbox{            }....  \mbox{  }a_{ {\bf{p}} - {\bf{q}}_2/2 }({\bf{q}}_2;t)\mbox{           }  = \mbox{             } 
 e^{ -i \epsilon_{ {\bf{p}}-{\bf{q}}_2 } t } e^{ - i \frac{({\bf{p}} - {\bf{q}}_2/2).{\bf{q}}_2}{m} t }  \mbox{             } 
  c_{ {\bf{p}}-{\bf{q}}_2,< } \mbox{            }....  \mbox{  }a_{ {\bf{p}} - {\bf{q}}_2/2 }({\bf{q}}_2)
\]
\[
  \mbox{           }  = \mbox{             } 
 e^{ -i \frac{ ({\bf{p}}-{\bf{q}}_2)^2 }{2m} t } e^{ - i \frac{({\bf{p}} - {\bf{q}}_2/2).{\bf{q}}_2}{m} t }  \mbox{             } 
  c_{ {\bf{p}}-{\bf{q}}_2,< } \mbox{            }....  \mbox{  }a_{ {\bf{p}} - {\bf{q}}_2/2 }({\bf{q}}_2)
    \mbox{           }  = \mbox{             } 
 e^{ -i \epsilon_p t }  \mbox{             } 
  c_{ {\bf{p}}-{\bf{q}}_2,< } \mbox{            }....  \mbox{  }a_{ {\bf{p}} - {\bf{q}}_2/2 }({\bf{q}}_2)
\]
Similarly,
\[
  c^{\dagger}_{ {\bf{p}}+{\bf{q}}_1,> }(t) \mbox{          }.....\mbox{          }
 a_{ {\bf{p}}+ {\bf{q}}_1/2 }({\bf{q}}_1;t)\mbox{          }=\mbox{          }
e^{ i \epsilon_p t }  
 \mbox{   } 
   c^{\dagger}_{ {\bf{p}}+{\bf{q}}_1,> } \mbox{          }.....\mbox{          }
 a_{ {\bf{p}}+ {\bf{q}}_1/2 }({\bf{q}}_1)
\]
Thus this correspondence gives the correct time evolution. Next we show that this correspondence when back substituted into the equations of the previous section leads to an identity.

\section{ Back substitution }

Set,
 \[
e^{ i N \theta_{ {\bf{p}} } }e^{ i N \theta_{ {\bf{p}}^{'} } } \mbox{            } 
= \mbox{           } 0 
 \]
and
 \[
e^{ i N \theta_{ {\bf{p}} } }e^{ -i N \theta_{ {\bf{p}}^{'} } } \mbox{            } 
= \mbox{           } \delta_{ {\bf{p}}, {\bf{p}}^{'} }
 \]
 This means,
\[
c_{ {\bf{p}}-{\bf{q}}/2, > }  =  \sum_{ {\bf{q}}_2 }
 \mbox{     } e^{- i N \theta_{ {\bf{p}}-{\bf{q}}/2-{\bf{q}}_2 } }
 \mbox{        }
 c_{ {\bf{p}}-{\bf{q}}/2-{\bf{q}}_2,< } \mbox{            } \frac{1}{\sqrt{n_{{\bf{p}}-{\bf{q}}/2-{\bf{q}}_2,<}}}  \mbox{   }
  \frac{1}{\sqrt{N_{>}+1}}  \mbox{  }a_{ {\bf{p}}-{\bf{q}}/2 - {\bf{q}}_2/2 }({\bf{q}}_2)
\]
and
\[
c^{\dagger}_{ {\bf{p}}+{\bf{q}}/2, < }    \mbox{            }  =    \mbox{            }  \sqrt{N_{>}+1}\mbox{          }\frac{1}{\sqrt{n_{{\bf{p}}+{\bf{q}}/2,<}}} \mbox{        }
c^{\dagger}_{ {\bf{p}}+{\bf{q}}/2, < } \mbox{  } e^{ i N \theta_{ {\bf{p}}+{\bf{q}}/2 } }
\mbox{       } 
\]
\[
+ \sum_{  {\bf{q}}_1 } \frac{1}{\sqrt{n_{{\bf{p}}+{\bf{q}}/2+{\bf{q}}_1,>}}}\mbox{       } 
  c^{\dagger}_{ {\bf{p}}+{\bf{q}}/2+{\bf{q}}_1,> } \mbox{          }\frac{1}{\sqrt{N_{>}+1}}
\mbox{  } e^{- i N \theta_{ {\bf{p}}+{\bf{q}}/2+{\bf{q}}_1 } }\mbox{          }
 a_{ {\bf{p}}+{\bf{q}}/2+ {\bf{q}}_1/2 }({\bf{q}}_1)
\]

---------------------------------------------------------------------------------------------

\[
c^{\dagger}_{ {\bf{p}}+{\bf{q}}/2, > }  = \sum_{ {\bf{q}}_2 }
   \mbox{   }a^{\dagger}_{ {\bf{p}}+{\bf{q}}/2 - {\bf{q}}_2/2 }({\bf{q}}_2)\mbox{     } \frac{1}{\sqrt{N_{>}+1}}
 \mbox{            } \frac{1}{\sqrt{n_{{\bf{p}}+{\bf{q}}/2-{\bf{q}}_2,<}}} \mbox{        }
 c^{\dagger}_{ {\bf{p}}+{\bf{q}}/2-{\bf{q}}_2,< }\mbox{  } e^{ i N \theta_{ {\bf{p}}+{\bf{q}}/2-{\bf{q}}_2 } }  
\]
and
\[
c_{ {\bf{p}}-{\bf{q}}/2, < }    \mbox{            }  =  \mbox{          } e^{ -i N \theta_{ {\bf{p}}-{\bf{q}}/2 } } \mbox{        }
c_{ {\bf{p}}-{\bf{q}}/2, < }
\frac{1}{\sqrt{n_{{\bf{p}}-{\bf{q}}/2,<}}}  \mbox{            }  \sqrt{N_{>}+1}
\mbox{       } 
\]
\[
+ \sum_{  {\bf{q}}_1 }
   a^{\dagger}_{ {\bf{p}}-{\bf{q}}/2+ {\bf{q}}_1/2 }({\bf{q}}_1)\mbox{   }
    e^{ i N \theta_{ {\bf{p}}-{\bf{q}}/2+{\bf{q}}_1 } }\mbox{  }
\frac{1}{\sqrt{N_{>}+1}}\mbox{        }  c_{ {\bf{p}}-{\bf{q}}/2+{\bf{q}}_1,> } \mbox{          }  \frac{1}{\sqrt{n_{{\bf{p}}-{\bf{q}}/2+{\bf{q}}_1,>}}}
\]

=========================================================================================

{\bf{Off-diagonal}}: 

\[
c^{\dagger}_{ {\bf{p}}+{\bf{q}}/2, > }  c_{ {\bf{p}}-{\bf{q}}/2, < } 
\mbox{              } = \mbox{                }  \sum_{ {\bf{q}}_2 }
   \mbox{   }a^{\dagger}_{ {\bf{p}}+{\bf{q}}/2 - {\bf{q}}_2/2 }({\bf{q}}_2)
 \mbox{            }\frac{1}{\sqrt{N_{>}+1}} 
 \mbox{               } \frac{1}{\sqrt{n_{{\bf{p}}+{\bf{q}}/2-{\bf{q}}_2,<}}} \mbox{        }
 c^{\dagger}_{ {\bf{p}}+{\bf{q}}/2-{\bf{q}}_2,< }\mbox{  } e^{ i N \theta_{ {\bf{p}}+{\bf{q}}/2-{\bf{q}}_2 } }\mbox{     }   c_{ {\bf{p}}-{\bf{q}}/2, < }  
\]
\[
\mbox{              } = \mbox{                }  \sum_{ {\bf{q}}_2 }
   \mbox{   }a^{\dagger}_{ {\bf{p}}+{\bf{q}}/2 - {\bf{q}}_2/2 }({\bf{q}}_2)
 \mbox{            }  \frac{1}{\sqrt{N_{>}+1}} 
 \mbox{               } \frac{1}{\sqrt{n_{{\bf{p}}+{\bf{q}}/2-{\bf{q}}_2,<}}} \mbox{        }
 c^{\dagger}_{ {\bf{p}}+{\bf{q}}/2-{\bf{q}}_2,< }\mbox{  } e^{ i N \theta_{ {\bf{p}}+{\bf{q}}/2-{\bf{q}}_2 } }\mbox{     }
 \mbox{          } \mbox{  } e^{ -i N \theta_{ {\bf{p}}-{\bf{q}}/2 } } \mbox{        }
c_{ {\bf{p}}-{\bf{q}}/2, < }
\frac{1}{\sqrt{n_{{\bf{p}}-{\bf{q}}/2,<}}}  \mbox{            }  \sqrt{N_{>}+1}
\]
\[
\mbox{              } = 
   \mbox{   }a^{\dagger}_{ {\bf{p}} }({\bf{q}})
\]

and

\[
c^{\dagger}_{ {\bf{p}}+{\bf{q}}/2, < } c_{ {\bf{p}}-{\bf{q}}/2, >}   \mbox{            }  =    \mbox{            }    \sqrt{N_{>}+1}\mbox{          }
\frac{1}{\sqrt{n_{{\bf{p}}+{\bf{q}}/2,<}}} \mbox{        }
c^{\dagger}_{ {\bf{p}}+{\bf{q}}/2, < } \mbox{  } e^{ i N \theta_{ {\bf{p}}+{\bf{q}}/2 } }
\mbox{       } c_{ {\bf{p}}-{\bf{q}}/2, >}  
\]
\[
+ \mbox{          }\sum_{  {\bf{q}}_1 } \frac{1}{\sqrt{n_{{\bf{p}}+{\bf{q}}/2+{\bf{q}}_1,>}}}\mbox{       } 
  c^{\dagger}_{ {\bf{p}}+{\bf{q}}/2+{\bf{q}}_1,> } \frac{1}{\sqrt{N_{>}+1}}
\mbox{  } e^{- i N \theta_{ {\bf{p}}+{\bf{q}}/2+{\bf{q}}_1 } }\mbox{          }
 a_{ {\bf{p}}+{\bf{q}}/2+ {\bf{q}}_1/2 }({\bf{q}}_1)\mbox{   }c_{ {\bf{p}}-{\bf{q}}/2, >} 
\]
\[
  \mbox{            }  =    \mbox{            } \sqrt{N_{>}+1}\mbox{          }
\frac{1}{\sqrt{n_{{\bf{p}}+{\bf{q}}/2,<}}} \mbox{        }
c^{\dagger}_{ {\bf{p}}+{\bf{q}}/2, < } \mbox{  } e^{ i N \theta_{ {\bf{p}}+{\bf{q}}/2 } }
\mbox{       }  \sum_{ {\bf{q}}_2 }
   \mbox{  } e^{- i N \theta_{ {\bf{p}}-{\bf{q}}/2-{\bf{q}}_2 } }
 \mbox{        }
 c_{ {\bf{p}}-{\bf{q}}/2-{\bf{q}}_2,< } \mbox{            } \frac{1}{\sqrt{n_{{\bf{p}}-{\bf{q}}/2-{\bf{q}}_2,<}}} \mbox{     } \frac{1}{\sqrt{N_{>}+1}} \mbox{   }a_{ {\bf{p}}-{\bf{q}}/2 - {\bf{q}}_2/2 }({\bf{q}}_2)
\]
\[
\mbox{            }  =   \mbox{   }a_{ {\bf{p}} }(-{\bf{q}})
\]
 


{\bf{Diagonal}}: 

\[
c^{\dagger}_{ {\bf{p}}+{\bf{q}}/2, > }
c_{ {\bf{p}}-{\bf{q}}/2, > } \mbox{              }  = 
\mbox{                   } 
\]
\[ \sum_{ {\bf{q}}_1,{\bf{q}}_2 }
   \mbox{   }a^{\dagger}_{ {\bf{p}}+{\bf{q}}/2 - {\bf{q}}_1/2 }({\bf{q}}_1)\mbox{     } \frac{1}{\sqrt{N_{>}+1}}
 \mbox{            } \frac{1}{\sqrt{n_{{\bf{p}}+{\bf{q}}/2-{\bf{q}}_1,<}}} \mbox{        }
 c^{\dagger}_{ {\bf{p}}+{\bf{q}}/2-{\bf{q}}_1,< }\mbox{  } e^{ i N \theta_{ {\bf{p}}+{\bf{q}}/2-{\bf{q}}_1 } }  
\] 
\[
 \mbox{     } e^{- i N \theta_{ {\bf{p}}-{\bf{q}}/2-{\bf{q}}_2 } }
 \mbox{        }
 c_{ {\bf{p}}-{\bf{q}}/2-{\bf{q}}_2,< } \mbox{            } \frac{1}{\sqrt{n_{{\bf{p}}-{\bf{q}}/2-{\bf{q}}_2,<}}}  \mbox{   }
  \frac{1}{\sqrt{N_{>}+1}}  \mbox{  }a_{ {\bf{p}}-{\bf{q}}/2 - {\bf{q}}_2/2 }({\bf{q}}_2)
\]
\[
\mbox{              }  = 
\mbox{                   } 
  \sum_{ {\bf{q}}_1 }
   \mbox{   }a^{\dagger}_{ {\bf{p}}+{\bf{q}}/2 - {\bf{q}}_1/2 }({\bf{q}}_1)\mbox{     } \frac{1}{N_{>}+1}  \mbox{  }a_{ {\bf{p}} - {\bf{q}}_1/2 }(-{\bf{q}}+{\bf{q}}_1)
\]
and,
\[
c_{ {\bf{p}}-{\bf{q}}/2, < } c^{\dagger}_{ {\bf{p}}+{\bf{q}}/2, < }    \mbox{            }  =  \mbox{          }  
\]
\[
e^{ -i N \theta_{ {\bf{p}}-{\bf{q}}/2 } } \mbox{        }
c_{ {\bf{p}}-{\bf{q}}/2, < }
\frac{1}{\sqrt{n_{{\bf{p}}-{\bf{q}}/2,<}}}  \mbox{            }  \sqrt{N_{>}+1}
 \mbox{           } \sqrt{N_{>}+1}\mbox{          }\frac{1}{\sqrt{n_{{\bf{p}}+{\bf{q}}/2,<}}} \mbox{        }
c^{\dagger}_{ {\bf{p}}+{\bf{q}}/2, < } \mbox{  } e^{ i N \theta_{ {\bf{p}}+{\bf{q}}/2 } }
\mbox{       } 
\]
\[
+ \sum_{  {\bf{q}}_1 }
   a^{\dagger}_{ {\bf{p}}-{\bf{q}}/2+ {\bf{q}}_1/2 }({\bf{q}}_1)\mbox{   }
    e^{ i N \theta_{ {\bf{p}}-{\bf{q}}/2+{\bf{q}}_1 } }\mbox{  }
\frac{1}{\sqrt{N_{>}+1}}\mbox{        }  c_{ {\bf{p}}-{\bf{q}}/2+{\bf{q}}_1,> } \mbox{          }  \frac{1}{\sqrt{n_{{\bf{p}}-{\bf{q}}/2+{\bf{q}}_1,>}}}
\]
\[
\times \mbox{             } 
  \sum_{  {\bf{q}}_2 } \frac{1}{\sqrt{n_{{\bf{p}}+{\bf{q}}/2+{\bf{q}}_2,>}}}\mbox{       } 
  c^{\dagger}_{ {\bf{p}}+{\bf{q}}/2+{\bf{q}}_2,> } \mbox{          }\frac{1}{\sqrt{N_{>}+1}}
\mbox{  } e^{- i N \theta_{ {\bf{p}}+{\bf{q}}/2+{\bf{q}}_2 } }\mbox{          }
 a_{ {\bf{p}}+{\bf{q}}/2+ {\bf{q}}_2/2 }({\bf{q}}_2)
\]
\[
   \mbox{            }  =  \mbox{          }   \sum_{  {\bf{q}}_1 }
   a^{\dagger}_{ {\bf{p}}-{\bf{q}}/2+ {\bf{q}}_1/2 }({\bf{q}}_1)\mbox{   }
\frac{1}{N_{>}+1}\mbox{          }
 a_{ {\bf{p}}+ {\bf{q}}_1/2 }(-{\bf{q}}+{\bf{q}}_1)
\]
\\ \mbox{  } \\
{\bf{ Thus, we have verified that the correspondence for the Fermi fields in terms of $ a_{ {\bf{k}} }({\bf{q}}) $ is fully correct. }}
\section{Solving the implicit equations}
We want to write $c_{ {\bf{p}}-{\bf{q}}/2, > }$, $c^{\dagger}_{ {\bf{p}}+{\bf{q}}/2, < }$, $c^{\dagger}_{ {\bf{p}}+{\bf{q}}/2, > }$ and $c_{ {\bf{p}}-{\bf{q}}/2, < }    \mbox{            } $ in terms of $a_{ {\bf{p}} }({\bf{q}})$.
For this, operator, which was implicit in the expansion of $c_{\bf{p},>}$ and the first term of $c_{\bf{p},<}$ in terms of $ a_{ {\bf{k}} }({\bf{q}}) $ is,
\begin{equation}\label{q<}
  q(\boldsymbol{p}-\boldsymbol{q_{2}},<) =c_{ {\boldsymbol{p}}-{\boldsymbol{q}}_2,< } \mbox{            } \frac{1}{\sqrt{n_{{\boldsymbol{p}}-{\boldsymbol{q}}_2,<}}}  \mbox{   }
\end{equation}
and the operator implicit in $c^{\dagger}_{\bf{p},>}$ and first term of $c^{\dagger}_{\bf{p},<}$ is,
\begin{equation}\label{q^+<}
   q^{\dagger}(\boldsymbol{p}-\boldsymbol{q_{2}},<)=\frac{1}{\sqrt{n_{{\boldsymbol{p}}-{\boldsymbol{q}_{2}},<}}} \mbox{        }
c^{\dagger}_{ {\boldsymbol{p}}-{\boldsymbol{q}_{2}}, < } .
\end{equation}
We call the operator, which was implicit in $c^{\dagger}_{\bf{p},<}$ ,
\begin{equation}\label{q^+>}
    q^{\dagger}(\boldsymbol{p}+\boldsymbol{q_{1}},>)=\frac{1}{\sqrt{n_{{\boldsymbol{p}}+{\boldsymbol{q}_{1}},>}}} \mbox{        }
c^{\dagger}_{ {\boldsymbol{p}}+{\boldsymbol{q}_{1}}, > }
\end{equation}
and the operator which was implicit in $c_{\bf{p},<}$ is,
\begin{equation}\label{q>}
q(\boldsymbol{p}+\boldsymbol{q_{1}},>) =c_{ {\boldsymbol{p}}+{\boldsymbol{q}}_1,> } \mbox{            } \frac{1}{\sqrt{n_{{\boldsymbol{p}}+{\boldsymbol{q}}_1,>}}}  \mbox{   } 
\end{equation}

Here, $\bf{p}$ is the momentum associated with the creation or annihilation operator, $\bf{q}_1$ and $\bf{q}_2$ are the summing indices.\\
Note:-\\
We define that the action of $q(\bf{p},< ~or~>)$ on a state that doesn't have a particle with momentum, $\bf{p}$ gives us 0.\\
\subsection{Basic equations}
\subsubsection{For q lesser than}
We want to express $q(\boldsymbol{p}-\boldsymbol{q_{2}},<)q^{\dagger}(\boldsymbol{p}-\boldsymbol{q_{2}},<)$ in terms of $n_{{\boldsymbol{p}}-{\boldsymbol{q}_{2}},<}$ and further express $n_{{\boldsymbol{p}}-{\boldsymbol{q}_{2}},<}$ in terms of  $ a_{ {\bf{k}} }({\bf{q}}) $. Since $c^{\dagger}_{ {\boldsymbol{p}}-{\boldsymbol{q}_{2}}, < }$  creates a particle  with momentum $|\boldsymbol{p}-\boldsymbol{q_{2}}|<k_{F} $ increasing the eigenvalue of $n_{{\boldsymbol{p}}-{\boldsymbol{q}_{2}},<}$ by 1,  we can re-write as:

\[q(\boldsymbol{p}-\boldsymbol{q_{2}},<)q^{\dagger}(\boldsymbol{p}-\boldsymbol{q_{2}},<)=c_{ {\boldsymbol{p}}-{\boldsymbol{q}}_2,< } \mbox{            } \frac{1}{\sqrt{n_{{\boldsymbol{p}}-{\boldsymbol{q}}_2,<}}} \times c^{\dagger}_{ {\boldsymbol{p}}-{\boldsymbol{q}_{2}}, <} \frac{1}{\sqrt{n_{{\boldsymbol{p}}-{\boldsymbol{q}_{2}},<}+1}}\]
Substituting the anti-commutation relation, we get,
\begin{equation}\label{q<q<^+}
    q(\boldsymbol{p}-\boldsymbol{q_{2}},<)q^{\dagger}(\boldsymbol{p}-\boldsymbol{q_{2}},<)=(1-n_{{\boldsymbol{p}}-{\boldsymbol{q}_{2}},<})(\frac{1}{(1+n_{{\boldsymbol{p}}-{\boldsymbol{q}_{2}},<})})
\end{equation}
On the other hand, we get, 
\begin{equation}\label{q<^+q<}
    q^{\dagger}(\boldsymbol{p}-\boldsymbol{q_{2}},<)q(\boldsymbol{p}-\boldsymbol{q_{2}},<)=1
\end{equation}
Since the results we obtained are in terms of  $n_{{\boldsymbol{p}}-{\boldsymbol{q}_{2}},<}$, we can use the relation between $ a_{ {\bf{k}} }({\bf{q}}) $ and $n_{{\boldsymbol{p}}-{\boldsymbol{q}_{2}},<}$,
\[
n_{{\boldsymbol{p}}-{\boldsymbol{q}_{2}},<}=c^{\dagger}_{ {\bf{p}-\bf{q_{2}}}, < }c_{ {\bf{p}-\bf{q_{2}}}, < }  \mbox{     }  = n_F(\bf{p}-\bf{q_{2}})\mbox{           } 
 - \sum_{ {\bf{q}}_1 }  a^{\dagger}_{ {\bf{p}-\bf{q_{2}}}+ {\bf{q}}_1/2 }({\bf{q}}_1)\mbox{  }
  \frac{1}{N_{>}+1} \mbox{  }
a_ {\bf{p-q_{2}}+{\bf{q}}_1/2 }({\bf{q}}_1)
\]
\subsubsection{For q greater than:}
\[q(\boldsymbol{p}+\boldsymbol{q_{1}},>) q^{\dagger}(\boldsymbol{p}+\boldsymbol{q_{1}},>)=c_{ {\boldsymbol{p}}+{\boldsymbol{q}}_1,> } \mbox{            } \frac{1}{\sqrt{n_{{\boldsymbol{p}}+{\boldsymbol{q}}_1,>}}}  \mbox{   }\frac{1}{\sqrt{n_{{\boldsymbol{p}}+{\boldsymbol{q}_{1}},>}}} \mbox{        }
c^{\dagger}_{ {\boldsymbol{p}}+{\boldsymbol{q}_{1}}, > } \]
Hence, we get,
\begin{equation}\label{q>q>^+}
    q(\boldsymbol{p}+\boldsymbol{q_{1}},>) q^{\dagger}(\boldsymbol{p}+\boldsymbol{q_{1}},>)= (1-n_{{\boldsymbol{p}}-{\boldsymbol{q}_{2}},>})\frac{1}{(1+n_{{\boldsymbol{p}}-{\boldsymbol{q}_{2}},>})}
\end{equation}
Similar to the previous result, we get,
\begin{equation}\label{q>^+q>}
    q^{\dagger}(\boldsymbol{p}+\boldsymbol{q_{1}},>)q(\boldsymbol{p}+\boldsymbol{q_{1}},>) =1
\end{equation}
\subsubsection{Results and summary so far}
\begin{enumerate}
    \item It is unclear how to decompose and find the qs from these equations, which are in terms of $n_{\bf{p}}$ as a function of a sum of $a^{\dagger}a$.
    \item By looking at Eq. \eqref{q<^+q<} and Eq. \eqref{q>^+q>} it seems like a unitary operation is applied to $a_{\bf{k}}(\bf{q})$ to get $c_{\bf{p}}$ in term of $a_{\bf{k}}(\bf{q})$. 
    \item  In addition to Eq. \eqref{q<q<^+} , Eq. \eqref{q>q>^+} and Eq. \eqref{q<^+q<} , Eq. \eqref{q>^+q>},  it seems like we need to find an equation that links qs to an operator that helps in jumping from the one tensor product Hilbert space to another tensor product Hilbert space of different dimension in the Fock space (some operator like $X_0$). Maybe  we can then compare terms and determine the momentum dependence.
\end{enumerate}
\subsection{Integral equation for $ q $ }
Set ($ s \mbox{             } = \mbox{                }  >,< $),
\[
q_{ {\bf{p}}, s } \equiv c_{ {\bf{p}}, s }\mbox{          }  \frac{1}{\sqrt{n_{ {\bf{p}}, s } }}
\]
so that,
\[
q_{ {\bf{p}}, > } \mbox{   }\sqrt{n_{ {\bf{p}}, > } }
\mbox{               }   = \mbox{                 }   \sum_{ {\bf{q}}_2 }
 \mbox{     } e^{- i N \theta_{ {\bf{p}}-{\bf{q}}_2 } }
 \mbox{        }q_{ {\bf{p}}-{\bf{q}}_2,< }  \mbox{   }
  \frac{1}{\sqrt{N_{>}+1}}  \mbox{  }a_{ {\bf{p}} - {\bf{q}}_2/2 }({\bf{q}}_2)
\]
and
\[
\sqrt{ n_{ {\bf{p}}, < } }\mbox{             } q^{\dagger}_{ {\bf{p}}, < }    \mbox{            }  =    \mbox{            } \sqrt{N_{>}+1}\mbox{          }q^{\dagger}_{ {\bf{p}}, < }\mbox{  } e^{ i N \theta_{ {\bf{p}} } }
\mbox{       } 
+ \sum_{  {\bf{q}}_1 }   q^{\dagger}_{ {\bf{p}}+{\bf{q}}_1,> } \mbox{          }\frac{1}{\sqrt{N_{>}+1}}
\mbox{  } e^{- i N \theta_{ {\bf{p}}+{\bf{q}}_1 } }\mbox{          }
 a_{ {\bf{p}}+ {\bf{q}}_1/2 }({\bf{q}}_1)
\]
We now have to solve these,
\[
 q_{ {\bf{p}}, < }
   \mbox{            }  =    \mbox{            }  \sum_{  {\bf{q}}_1 }  
\mbox{          }
 a^{\dagger}_{ {\bf{p}}+ {\bf{q}}_1/2 }({\bf{q}}_1)\mbox{  } e^{ i N \theta_{ {\bf{p}}+{\bf{q}}_1 } }\mbox{          }\frac{1}{\sqrt{N_{>}+1}}\mbox{             }q_{ {\bf{p}}+{\bf{q}}_1,> } \mbox{            } 
 \left( \sqrt{ n_{ {\bf{p}}, < } }    \mbox{            }  -   \mbox{            } \sqrt{N_{>}+1}\mbox{  } e^{ -i N \theta_{ {\bf{p}} } }\right)^{-1}
\]
and
\[
q_{ {\bf{p}}, > } \mbox{   }\sqrt{n_{ {\bf{p}}, > } }
\mbox{               }   = \mbox{                 }   \sum_{ {\bf{q}}_2 }
 \mbox{     } e^{- i N \theta_{ {\bf{p}}-{\bf{q}}_2 } }
 \mbox{        }q_{ {\bf{p}}-{\bf{q}}_2,< }  \mbox{   }
  \frac{1}{\sqrt{N_{>}+1}}  \mbox{  }a_{ {\bf{p}} - {\bf{q}}_2/2 }({\bf{q}}_2)
\]

=======================================================================================

\[
q_{ {\bf{p}}+{\bf{q}}_1, > }
\mbox{               }   = \mbox{                 }   \sum_{ {\bf{q}}_2 }
 \mbox{     } e^{- i N \theta_{ {\bf{p}}+{\bf{q}}_1-{\bf{q}}_2 } }
 \mbox{        }q_{ {\bf{p}}+{\bf{q}}_1-{\bf{q}}_2,< }  \mbox{   }
  \frac{1}{\sqrt{N_{>}+1}}  \mbox{  }a_{ {\bf{p}}+{\bf{q}}_1 - {\bf{q}}_2/2 }({\bf{q}}_2) \mbox{   }\frac{1}{\sqrt{n_{ {\bf{p}}+{\bf{q}}_1, > } }}
\]

Thus we get the following integral equation:

\[
 q_{ {\bf{p}}, < }
   \mbox{            }  =    \mbox{            }  \sum_{  {\bf{q}}_1, {\bf{q}}_2 }
\mbox{          }
 a^{\dagger}_{ {\bf{p}}+ {\bf{q}}_1/2 }({\bf{q}}_1)\mbox{  }\frac{ e^{ i N \theta_{ {\bf{p}}+{\bf{q}}_1 } }\mbox{          } e^{- i N \theta_{ {\bf{p}}+{\bf{q}}_1-{\bf{q}}_2 } }}{\sqrt{N_{>}+1}}
 \mbox{        }q_{ {\bf{p}}+{\bf{q}}_1-{\bf{q}}_2,< }  \mbox{   }
 \]\[\frac{1}{\sqrt{N_{>}+1}}  \mbox{  }a_{ {\bf{p}}+{\bf{q}}_1 - {\bf{q}}_2/2 }({\bf{q}}_2) \mbox{   }\frac{1}{\sqrt{n_{ {\bf{p}}+{\bf{q}}_1, > } }}\mbox{            } 
   \left( \sqrt{ n_{ {\bf{p}}, < } }    \mbox{            }  -   \mbox{            } \sqrt{N_{>}+1}\mbox{  } e^{ -i N \theta_{ {\bf{p}} } }\right)^{-1}
\]
Taking $\bf{q}_{1}-\bf{q}_{2}=\bf{Q}$ and $\frac{\bf{q}_{1}+\bf{q}_{2}}{2}=\bf{K}$ which gives us $\bf{q}_1=\bf{K}+\bf{Q}/2$ and $\bf{q}_2=\bf{K}-\bf{Q}/2$
\[
 q_{ {\bf{p}}, < }
   \mbox{            }  =    \mbox{            }  \sum_{ {\bf{K}}, {\bf{Q}} }
\mbox{          }
 a^{\dagger}_{ {\bf{p}}+ ({\bf{K}}+{\bf{Q}}/2)/2 }({\bf{K}}+{\bf{Q}}/2)\mbox{  }
 \frac{ e^{ i N \theta_{ {\bf{p}}+{\bf{K}}+{\bf{Q}}/2 } }\mbox{          } e^{- i N \theta_{ {\bf{p}}+{\bf{Q}} } }}{\sqrt{N_{>}+1}}
 \mbox{        }q_{ {\bf{p}}+{\bf{Q}} ,< }  \mbox{   }
\]
\[  \frac{1}{\sqrt{N_{>}+1}}  \mbox{  }a_{ {\bf{p}}+{\bf{K}}+{\bf{Q}}/2 - ({\bf{K}}-{\bf{Q}}/2)/2 }({\bf{K}}-{\bf{Q}}/2) 
  \mbox{   }\frac{1}{\sqrt{n_{ {\bf{p}}+{\bf{K}}+{\bf{Q}}/2 , > } }}\mbox{            }
   \left( \sqrt{ n_{ {\bf{p}}, < } }    \mbox{            }  -   \mbox{            } \sqrt{N_{>}+1}\mbox{  } e^{ -i N \theta_{ {\bf{p}} } }\right)^{-1}
\]
Simplifying and re-writing as:
\[
 q_{ {\bf{p}}, < }
   \mbox{            }  =    \mbox{            } \sum_ {\bf{K}}\sum_ {\bf{Q}}  
\mbox{          }
 a^{\dagger}_{ {\bf{p}}+ ({\bf{K}}+{\bf{Q}}/2)/2 }({\bf{K}}+{\bf{Q}}/2)\mbox{  }
 \frac{ e^{ i N \theta_{ {\bf{p}}+{\bf{K}}+{\bf{Q}}/2 } }\mbox{          } e^{- i N \theta_{ {\bf{p}}+{\bf{Q}} } }}{\sqrt{N_{>}+1}}
 \mbox{        }q_{ {\bf{p}}+{\bf{Q}} ,< }  \mbox{   }
\]
\[  \frac{1}{\sqrt{N_{>}+1}}  \mbox{  }a_{ {\bf{p}}+{\bf{K}/2}+{3\bf{Q}}/4 }({\bf{K}}-{\bf{Q}}/2) 
  \mbox{   }\frac{1}{\sqrt{n_{ {\bf{p}}+{\bf{K}}+{\bf{Q}}/2 , > } }}\mbox{            }
   \left( \sqrt{ n_{ {\bf{p}}, < } }    \mbox{            }  -   \mbox{            } \sqrt{N_{>}+1}\mbox{  } e^{ -i N \theta_{ {\bf{p}} } }\right)^{-1}
\]
First, let us define $C$ as:
\begin{equation}\label{C}
    C= \sum_ {\bf{K}}\sum_ {\bf{Q}}  
\mbox{          }
 a^{\dagger}_ {({\bf{K}}+{\bf{Q}}/2)/2 }({\bf{K}}+{\bf{Q}}/2)\mbox{  }
 \frac{ e^{ i N \theta_{ {\bf{K}}+{\bf{Q}}/2 } }\mbox{          } e^{- i N \theta_{{\bf{Q}} } }}{\sqrt{N_{>}+1}}
 \mbox{        }q_{ {\bf{Q}} ,< }  \mbox{   }\frac{1}{\sqrt{N_{>}+1}}  \mbox{  }a_{ {\bf{K}/2}+{3\bf{Q}}/4 }({\bf{K}}-{\bf{Q}}/2) 
  \mbox{   }\frac{1}{\sqrt{n_{{\bf{K}}+{\bf{Q}}/2 , > } }}
\end{equation}
To make $q_{ {\bf{p}}, < }$ explicit, we are going to define an operator $T(p)$ which behaves the following way: 
\[  T(\boldsymbol{p})\mbox{   }C= \sum_ {\bf{K}}\sum_ {\bf{Q}}  
\mbox{          }
 a^{\dagger}_{ {\bf{p}}+ ({\bf{K}}+{\bf{Q}}/2)/2 }({\bf{K}}+{\bf{Q}}/2)\mbox{  }
 \frac{ e^{ i N \theta_{ {\bf{p}}+{\bf{K}}+{\bf{Q}}/2 } }\mbox{          } e^{- i N \theta_{ {\bf{p}}+{\bf{Q}} } }}{\sqrt{N_{>}+1}}
 \mbox{        }q_{ {\bf{p}}+{\bf{Q}} ,< }  \mbox{   }\] 
In other words, $T(\bf{p})$ shifts the index of   $a_{f(\bf{K},\bf{Q})}(g(\bf{K},\bf{Q})$ , $\theta_{h(\bf{K},\bf{Q})}$ and $n_{(\bf{K},\bf{Q})}$ by $\bf{p}$ in the sum over $K$ and $Q$ in the above equation. 
We can re-write $q_{ {\bf{p}}, < }
   \mbox{            }$ as:    
   \[ q_{ {\bf{p}}, < }
   \mbox{            }= T(\boldsymbol{p})\mbox{   }C\left( \sqrt{ n_{ 0, < } }    \mbox{            }  -   \mbox{            } \sqrt{N_{>}+1}\mbox{  } e^{ -i N \theta_{ 0 } }\right)^{-1}
   \]
   We want to substitute for the $q_{ {\bf{Q}}, < }
   \mbox{            }$ in the definition of C.
   \begin{equation}\label{q<=TC}
       q_{ {\bf{Q}}, < }
   \mbox{            }= T(\boldsymbol{Q})\mbox{   }C\left( \sqrt{ n_{ 0, < } }    \mbox{            }  -   \mbox{            } \sqrt{N_{>}+1}\mbox{  } e^{ -i N \theta_{ 0 } }\right)^{-1}
   \end{equation}
   Substituting Eq. \eqref{q<=TC} in Eq. \eqref{C} we get:
   \[C= \sum_ {\bf{K}}\sum_ {\bf{Q}}  
\mbox{          }
 a^{\dagger}_ {({\bf{K}}+{\bf{Q}}/2)/2 }({\bf{K}}+{\bf{Q}}/2)\mbox{  }
 \frac{ e^{ i N \theta_{ {\bf{K}}+{\bf{Q}}/2 } }\mbox{          } e^{- i N \theta_{{\bf{Q}} } }}{\sqrt{N_{>}+1}}\]
 \[
 \mbox{        }T(\boldsymbol{Q})\mbox{   }C\left( \sqrt{ n_{ 0, < } }    \mbox{            }  -   \mbox{            } \sqrt{N_{>}+1}\mbox{  } e^{ -i N \theta_{ 0 } }\right)^{-1} \mbox{   }\frac{1}{\sqrt{N_{>}+1}}  \mbox{  }a_{ {\bf{K}/2}+{3\bf{Q}}/4 }({\bf{K}}-{\bf{Q}}/2) 
  \mbox{   }\frac{1}{\sqrt{n_{{\bf{K}}+{\bf{Q}}/2 , > } }}\]
   The above equation should be more easily solvable as the implicit variable $C$ is not summed over.
   The next idea is to find what T(Q) is from this equation. We can then substitute with T(Q) in Eq. \eqref{q<=TC}. 

\section{ Momentum distribution of interacting systems }

\[
H = \sum_{ {\bf{p}} }\epsilon_p \mbox{   } c^{\dagger}_{ {\bf{p}} }c_{ {\bf{p}} }
  + \sum_{ {\bf{q}} } \frac{v_q}{2V} (a_{.}(-{\bf{q}}) + a^{\dagger}_{.}({\bf{q}}))(a_{.}({\bf{q}}) + a^{\dagger}_{.}(-{\bf{q}}))
\]



\end{document} 